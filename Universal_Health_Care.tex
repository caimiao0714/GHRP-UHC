\documentclass[]{elsarticle} %review=doublespace preprint=single 5p=2 column
%%% Begin My package additions %%%%%%%%%%%%%%%%%%%
\usepackage[hyphens]{url}

  \journal{Global Health Research and Policy - Thematic Series: Universal Health Coverage} % Sets Journal name


\usepackage{lineno} % add
\providecommand{\tightlist}{%
  \setlength{\itemsep}{0pt}\setlength{\parskip}{0pt}}

\bibliographystyle{elsarticle-harv}
\biboptions{sort&compress} % For natbib
\usepackage{graphicx}
\usepackage{booktabs} % book-quality tables
%%%%%%%%%%%%%%%% end my additions to header

\usepackage[T1]{fontenc}
\usepackage{lmodern}
\usepackage{amssymb,amsmath}
\usepackage{ifxetex,ifluatex}
\usepackage{fixltx2e} % provides \textsubscript
% use upquote if available, for straight quotes in verbatim environments
\IfFileExists{upquote.sty}{\usepackage{upquote}}{}
\ifnum 0\ifxetex 1\fi\ifluatex 1\fi=0 % if pdftex
  \usepackage[utf8]{inputenc}
\else % if luatex or xelatex
  \usepackage{fontspec}
  \ifxetex
    \usepackage{xltxtra,xunicode}
  \fi
  \defaultfontfeatures{Mapping=tex-text,Scale=MatchLowercase}
  \newcommand{\euro}{€}
\fi
% use microtype if available
\IfFileExists{microtype.sty}{\usepackage{microtype}}{}
\usepackage[left=3cm,right=3cm,top=3cm,bottom=3cm]{geometry}
\usepackage{longtable}
\usepackage{graphicx}
% We will generate all images so they have a width \maxwidth. This means
% that they will get their normal width if they fit onto the page, but
% are scaled down if they would overflow the margins.
\makeatletter
\def\maxwidth{\ifdim\Gin@nat@width>\linewidth\linewidth
\else\Gin@nat@width\fi}
\makeatother
\let\Oldincludegraphics\includegraphics
\renewcommand{\includegraphics}[1]{\Oldincludegraphics[width=\maxwidth]{#1}}
\ifxetex
  \usepackage[setpagesize=false, % page size defined by xetex
              unicode=false, % unicode breaks when used with xetex
              xetex]{hyperref}
\else
  \usepackage[unicode=true]{hyperref}
\fi
\hypersetup{breaklinks=true,
            bookmarks=true,
            pdfauthor={},
            pdftitle={Association between compulsory health insurance and life expectency in 184 countries: A retrospective longitudinal study},
            colorlinks=true,
            urlcolor=blue,
            linkcolor=blue,
            pdfborder={0 0 0}}
\urlstyle{same}  % don't use monospace font for urls

\setcounter{secnumdepth}{5}
% Pandoc toggle for numbering sections (defaults to be off)
% Pandoc header
\usepackage{soulutf8}
\usepackage{color}
\usepackage{float}
\usepackage{booktabs}
\usepackage{lscape}
\usepackage{setspace}
\usepackage{dcolumn}



\begin{document}
\begin{frontmatter}

  \title{Association between compulsory health insurance and life expectency in 184 countries: A retrospective longitudinal study}
    \author[SLU]{Miao Cai}
   \ead{miao.cai@slu.edu} 
  
    \author[SLU]{Asabe Garba}
   \ead{asabe.garba@slu.edu} 
  
    \author[WHU]{Xin Li}
   \ead{xl60@iu.edu} 
  
    \author[SCU]{Xiaojun Lin\corref{c1}}
   \ead{xjlin@hust.edu.cn} 
   \cortext[c1]{Corresponding Author}
    \author[SLU]{Ziqi Peng}
   \ead{ziqi.peng@slu.edu} 
  
      \address[SLU]{College for Public Health and Social Justice, Saint Louis University, Saint Louis, MO, 63108}
    \address[SCU]{West China School of Public Health, Sichuan University, Chengdu, Sichuan, China, 610044}
    \address[WHU]{School of Information Management, Wuhan University, Wuhan, Hubei, China, 430072}
  
  \begin{abstract}
  This is the abstract.
  
  It consists of two paragraphs.
  \end{abstract}
  
 \end{frontmatter}

\newcommand{\blandscape}{\begin{landscape}}
\newcommand{\elandscape}{\end{landscape}}
\doublespacing

\hypertarget{introduction}{%
\section{Introduction}\label{introduction}}

\hl{government health expenditure and life expectancy.}

{[}\protect\hyperlink{ref-wagstaff2018progress}{1}{]}.

\hypertarget{methods}{%
\section{Methods}\label{methods}}

\hypertarget{data-source}{%
\subsection{Data source}\label{data-source}}

We extracted country level data from the Global Health Expenditure Database on the World Health Organization (WHO) website {[}\protect\hyperlink{ref-WHOdata}{2}{]}.
This database provides detailed comparable health expenditure data in around 190 countries from 2000 to 2016.
These health expenditure variables include current health expenditure (CHE) decomposition: domestic government health expenditure, private health expenditure, and out-of-pocket (OOP) payment as percent of CHE;
financing arrangements decomposition: compulsory financing arrangements, government financing arrangements, compulsory health insurance, household OOP payment as percent of CHE; CHE and government health expenditure as percent of gross domestic product (GDP).

The Global Health Observatory data repository by the WHO provides global health priority data by country in multiple years {[}\protect\hyperlink{ref-worldbank}{3}{]}.
We obtained life expectancy, GDP, and population data in provided countries from 2000 to 2016.
Both databases are publicly available, with downloadable comma-separated values or Microsoft Excel files provided on the WHO website.
The two databases were then merged according to two common keys, country name and year.

\hypertarget{variable-selection}{%
\subsection{Variable selection}\label{variable-selection}}

The outcome variable in this study was the life expectancy at birth in a country in a specific year.
It reflects the overll mortality level of all age groups in a country in a given year.
Life expectancy is one of the most widely used measure of mortality and burden of disease in previous literature {[}\protect\hyperlink{ref-lee2012effect}{4}--\protect\hyperlink{ref-mathers2015causes}{7}{]}.

We used four sets of explanatory variables to predict life expectancy.
Country level general characteristics include population (in millions), year (2000 to 2015), and GDP (in billions).
The GDP data were reported in constant 2010 prices, which were adjusted for the effects of price inflation {[}\protect\hyperlink{ref-worldbankconstant}{8}{]}.
Current health expenditure (CHE) and government health expenditure (GGHE-D) as percent of GDP were used to account for the investment in healthcare in a country.
Compulsory financing arrangements and compulsory health insurance as percent of the CHE were two major sources of public financing arrangements.
Private health expenditure and OOP payment as percent of the CHE were two sources healthcare expenditure.

\hypertarget{statistical-analyses}{%
\subsection{Statistical Analyses}\label{statistical-analyses}}

We excluded observations with missing data in either the dependent variable or any of the explanatory variables (n = 294), which resulted in 2,975 complete observations (91\% of the original data) from 184 countries.

A pooled ordinary least square model were used to estimate the association between compulsory health insurance and life expectancy among the 183 countries. Since high and low income countries were characterized by different patterns of life expectancy and health financing schemes, we further conducted stratified analyses among the four income category countries to allow for potentially different patterns of association.

Point and interval estimates (95\% confidence intervals, 95\% CI), as well as the p-values, were reported for all independent variables.
A p-value less than 0.05 is viewed as statistically significant.
All data cleaning, visualization, statistical modelling, and reporting were performed using R 3.5.3 {[}\protect\hyperlink{ref-R353}{9}{]}.
In an effort to promote reproducible research, we have created a public GitHub repository to store all the data and R code we use to write this paper.
Interested readers can find them at \url{https://github.com/caimiao0714/GHRP-UHC}.

\hypertarget{results}{%
\section{Results}\label{results}}

\begin{figure}
\centering
\includegraphics{Figures/fig1.pdf}
\caption{\label{fig:fig1}Life expectancy in 139 countries from 2000 to 2015}
\end{figure}

Figure \ref{fig:fig1} demonstrates the trend of life expectancy in the 139 countries over time, with each line represents a country while a color stands for an income category. The life expectancy in the studied countries were generally increasing from 2010 to 2015. The most significant pattern in the plot was that life expectancy was strongly influenced by income group: the high income countries have the highest life expectancy, while the low income countries generally had the lowest life expectancy. It was also to be noted that the variance of life expectancy in low and low-mid income countries were much higher than that in up-mid and high income countries.

\begin{table}[t]

\caption{\label{tab:unnamed-chunk-2}Descrptive statistics (2000-2016)}
\centering
\resizebox{\linewidth}{!}{
\begin{tabular}{lllll}
\toprule
  & Low & Low-Mid & Up-Mid & High\\
\midrule
N & 459 & 830 & 826 & 860\\
Life Expectancy & 56.65 (5.59) & 65.45 (7.26) & 71.22 (5.71) & 77.82 (3.35)\\
Current health expenditure as percent of GDP & 6.15 (2.46) & 5.36 (2.35) & 5.74 (2.15) & 7.21 (2.70)\\
Government Health Expenditure as percent of GDP & 1.40 (0.74) & 2.26 (1.66) & 3.18 (1.64) & 5.09 (2.15)\\
Private health expenditure as percent CHE & 49.91 (18.30) & 47.83 (22.12) & 42.65 (17.53) & 29.39 (12.74)\\
\addlinespace
Out-of-pocket payment as percent of CHE & 44.73 (18.82) & 43.37 (21.81) & 35.00 (17.77) & 22.00 (10.76)\\
Compulsory financing arrangements as percent of CHE & 36.18 (14.99) & 46.49 (21.04) & 55.40 (17.48) & 66.01 (20.50)\\
Compulsory health insurance as percent of CHE & 1.35 (2.53) & 6.58 (9.78) & 18.06 (22.52) & 23.53 (29.31)\\
Population (millions) & 17.71 (19.03) & 54.24 (172.60) & 48.99 (188.88) & 22.34 (47.24)\\
GDP & 0.90 (0.86) & 8.54 (24.51) & 28.61 (89.20) & 86.51 (223.73)\\
\bottomrule
\end{tabular}}
\end{table}

\hl{This is not a panel data. Are you sure that 139 countries from 2000 to 2015 were included in Figure 1? It seemed that over 170 countries were included; Data from 2016 was also included in the linear regression models}

\hl{Xiaojun Lin, create the Table 1 for f1. write results for Table 1, 2, 3 and Figure 1.}

\begin{table}[!htbp] \centering 
  \caption{OLS model predicting life expectancies in 139 countries from 2000 to 2015} 
  \label{poolOLS} 
\begin{tabular}{@{\extracolsep{5pt}}lc} 
\\[-1.8ex]\hline \\[-1.8ex] 
\\[-1.8ex] & Life expectancy \\ 
\hline \\[-1.8ex] 
 Current health expenditure as percent of GDP & 0.162$^{*}$ \\ 
  & (0.026, 0.298) \\ 
  Government health expenditure as percent of GDP & 0.482$^{***}$ \\ 
  & (0.263, 0.702) \\ 
  Private health expenditure as percent CHE & $-$0.154$^{***}$ \\ 
  & ($-$0.186, $-$0.122) \\ 
  Out-of-pocket payment as percent of CHE & 0.174$^{***}$ \\ 
  & (0.146, 0.201) \\ 
  Compulsory financing arrangements as percent of CHE & 0.0003 \\ 
  & ($-$0.016, 0.017) \\ 
  Compulsory health insurance as percent of CHE & 0.035$^{***}$ \\ 
  & (0.025, 0.045) \\ 
  Population (millions) & 0.002$^{**}$ \\ 
  & (0.001, 0.004) \\ 
  GDP & 0.001 \\ 
  & ($-$0.0003, 0.003) \\ 
  Year & 0.301$^{***}$ \\ 
  & (0.263, 0.339) \\ 
  Low income country & $-$19.084$^{***}$ \\ 
  & ($-$19.889, $-$18.280) \\ 
  Low to middle income country & $-$10.936$^{***}$ \\ 
  & ($-$11.539, $-$10.334) \\ 
  Up to middel income country & $-$5.414$^{***}$ \\ 
  & ($-$5.945, $-$4.882) \\ 
  Constant & $-$529.862$^{***}$ \\ 
  & ($-$605.762, $-$453.961) \\ 
 \textit{N} & 2,975 \\ 
R$^{2}$ & 0.695 \\ 
Adjusted R$^{2}$ & 0.694 \\ 
\hline \\[-1.8ex] 
\multicolumn{2}{l}{$^{*}$p $<$ .05; $^{**}$p $<$ .01; $^{***}$p $<$ .001} \\ 
\multicolumn{2}{l}{GDP: Gross Domestic Product} \\ 
\multicolumn{2}{l}{CHE: Current Health Expenditure} \\ 
\end{tabular} 
\end{table}

Table \ref{poolOLS} demonstrates XXXXXXX.

\begin{landscape}

\begin{table}[!htbp] \centering 
  \caption{OLS model predicting life expectancies from 2000 to 2015 stratifeid by country income categories} 
  \label{stratifiedOLS} 
\begin{tabular}{@{\extracolsep{5pt}}lD{.}{.}{-3} D{.}{.}{-3} D{.}{.}{-3} D{.}{.}{-3} } 
\\[-1.8ex]\hline \\[-1.8ex] 
\\[-1.8ex] & \multicolumn{4}{c}{Life expectancy} \\ 
 & \multicolumn{1}{c}{Low} & \multicolumn{1}{c}{Low-mid} & \multicolumn{1}{c}{Up-mid} & \multicolumn{1}{c}{High} \\ 
\\[-1.8ex] & \multicolumn{1}{c}{Model 1} & \multicolumn{1}{c}{Model 2} & \multicolumn{1}{c}{Model 3} & \multicolumn{1}{c}{Model 4}\\ 
\hline \\[-1.8ex] 
 Current Health Expenditure as percent of GDP & -0.155 & 0.157 & 1.022^{***} & -0.719^{***} \\ 
  & \multicolumn{1}{c}{(-0.377$, $0.067)} & \multicolumn{1}{c}{(-0.122$, $0.436)} & \multicolumn{1}{c}{(0.595$, $1.448)} & \multicolumn{1}{c}{(-1.039$, $-0.399)} \\ 
  Government Health Expenditure as percent of GDP & -0.394 & 0.099 & -0.890^{*} & 1.975^{***} \\ 
  & \multicolumn{1}{c}{(-1.189$, $0.402)} & \multicolumn{1}{c}{(-0.391$, $0.590)} & \multicolumn{1}{c}{(-1.597$, $-0.183)} & \multicolumn{1}{c}{(1.554$, $2.397)} \\ 
  Private Health Expenditure as percent CHE & 0.174^{***} & -0.373^{***} & -0.137^{***} & 0.116^{***} \\ 
  & \multicolumn{1}{c}{(0.087$, $0.262)} & \multicolumn{1}{c}{(-0.469$, $-0.277)} & \multicolumn{1}{c}{(-0.215$, $-0.059)} & \multicolumn{1}{c}{(0.072$, $0.159)} \\ 
  Out-of-pocket payment as percent of CHE & -0.172^{***} & 0.435^{***} & 0.239^{***} & -0.043^{*} \\ 
  & \multicolumn{1}{c}{(-0.254$, $-0.090)} & \multicolumn{1}{c}{(0.349$, $0.520)} & \multicolumn{1}{c}{(0.201$, $0.277)} & \multicolumn{1}{c}{(-0.076$, $-0.010)} \\ 
  Compulsory Financing Arrangements as percent of CHE & -0.024 & 0.054 & 0.148^{***} & 0.002 \\ 
  & \multicolumn{1}{c}{(-0.075$, $0.026)} & \multicolumn{1}{c}{(-0.008$, $0.116)} & \multicolumn{1}{c}{(0.087$, $0.210)} & \multicolumn{1}{c}{(-0.009$, $0.012)} \\ 
  Compulsory health insurance as percent of CHE & 0.224^{**} & 0.243^{***} & 0.061^{***} & -0.011^{***} \\ 
  & \multicolumn{1}{c}{(0.055$, $0.392)} & \multicolumn{1}{c}{(0.195$, $0.291)} & \multicolumn{1}{c}{(0.045$, $0.078)} & \multicolumn{1}{c}{(-0.018$, $-0.005)} \\ 
  Population (millions) & -0.056^{*} & -0.005 & 0.003 & 0.012 \\ 
  & \multicolumn{1}{c}{(-0.100$, $-0.012)} & \multicolumn{1}{c}{(-0.012$, $0.002)} & \multicolumn{1}{c}{(-0.001$, $0.007)} & \multicolumn{1}{c}{(-0.009$, $0.032)} \\ 
  GDP & 2.435^{***} & 0.060^{*} & 0.001 & -0.003 \\ 
  & \multicolumn{1}{c}{(1.427$, $3.444)} & \multicolumn{1}{c}{(0.011$, $0.109)} & \multicolumn{1}{c}{(-0.007$, $0.009)} & \multicolumn{1}{c}{(-0.007$, $0.002)} \\ 
  Year & 0.506^{***} & 0.289^{***} & 0.225^{***} & 0.158^{***} \\ 
  & \multicolumn{1}{c}{(0.409$, $0.602)} & \multicolumn{1}{c}{(0.200$, $0.378)} & \multicolumn{1}{c}{(0.160$, $0.290)} & \multicolumn{1}{c}{(0.122$, $0.195)} \\ 
  Constant & -958.790^{***} & -521.271^{***} & -395.563^{***} & -247.217^{***} \\ 
  & \multicolumn{1}{c}{(-1,152.821$, $-764.759)} & \multicolumn{1}{c}{(-699.343$, $-343.198)} & \multicolumn{1}{c}{(-525.258$, $-265.868)} & \multicolumn{1}{c}{(-319.848$, $-174.585)} \\ 
 \textit{N} & \multicolumn{1}{c}{459} & \multicolumn{1}{c}{830} & \multicolumn{1}{c}{826} & \multicolumn{1}{c}{860} \\ 
R$^{2}$ & \multicolumn{1}{c}{0.391} & \multicolumn{1}{c}{0.304} & \multicolumn{1}{c}{0.385} & \multicolumn{1}{c}{0.465} \\ 
Adjusted R$^{2}$ & \multicolumn{1}{c}{0.379} & \multicolumn{1}{c}{0.296} & \multicolumn{1}{c}{0.378} & \multicolumn{1}{c}{0.460} \\ 
\hline \\[-1.8ex] 
\multicolumn{5}{l}{$^{*}$p $<$ .05; $^{**}$p $<$ .01; $^{***}$p $<$ .001} \\ 
\end{tabular} 
\end{table}

\end{landscape}

Table \ref{stratifiedOLS} presents XXXXXXX.

\hypertarget{discussion}{%
\section{Discussion}\label{discussion}}

\hypertarget{acknowledgements}{%
\section*{Acknowledgements}\label{acknowledgements}}
\addcontentsline{toc}{section}{Acknowledgements}

We thank the WHO and the World Bank for making data used in this study publicly available for researchers.

\hypertarget{funding}{%
\section*{Funding}\label{funding}}
\addcontentsline{toc}{section}{Funding}

None.

\hypertarget{availability-of-data-and-materials}{%
\section*{Availability of data and materials}\label{availability-of-data-and-materials}}
\addcontentsline{toc}{section}{Availability of data and materials}

All data and associated R code are public available at the GitHub repository \texttt{caimiao0714/GHRP-UHC}, which can be accessed at \url{https://github.com/caimiao0714/GHRP-UHC}.

\hypertarget{references}{%
\section*{References}\label{references}}
\addcontentsline{toc}{section}{References}

\hypertarget{refs}{}
\leavevmode\hypertarget{ref-wagstaff2018progress}{}%
1. Wagstaff A, Flores G, Smitz M-F, Hsu J, Chepynoga K, Eozenou P. Progress on impoverishing health spending in 122 countries: A retrospective observational study. The Lancet Global Health. 2018;6:e180--92.

\leavevmode\hypertarget{ref-WHOdata}{}%
2. The World Health Organization. Global Health Expenditure Database. 2016. \url{http://apps.who.int/nha/database/Select/Indicators/en}. Accessed 20 Mar 2019.

\leavevmode\hypertarget{ref-worldbank}{}%
3. The World Bank. World Bank Open Data. 2018. \url{https://data.worldbank.org/}. Accessed 6 Apr 2018.

\leavevmode\hypertarget{ref-lee2012effect}{}%
4. Lee I-M, Shiroma EJ, Lobelo F, Puska P, Blair SN, Katzmarzyk PT, et al. Effect of physical inactivity on major non-communicable diseases worldwide: An analysis of burden of disease and life expectancy. The lancet. 2012;380:219--29.

\leavevmode\hypertarget{ref-salomon2012healthy}{}%
5. Salomon JA, Wang H, Freeman MK, Vos T, Flaxman AD, Lopez AD, et al. Healthy life expectancy for 187 countries, 1990--2010: A systematic analysis for the global burden disease study 2010. The Lancet. 2012;380:2144--62.

\leavevmode\hypertarget{ref-bennett2015future}{}%
6. Bennett JE, Li G, Foreman K, Best N, Kontis V, Pearson C, et al. The future of life expectancy and life expectancy inequalities in england and wales: Bayesian spatiotemporal forecasting. The Lancet. 2015;386:163--70.

\leavevmode\hypertarget{ref-mathers2015causes}{}%
7. Mathers CD, Stevens GA, Boerma T, White RA, Tobias MI. Causes of international increases in older age life expectancy. The Lancet. 2015;385:540--8.

\leavevmode\hypertarget{ref-worldbankconstant}{}%
8. The World Bank. What is the difference between current and constant data? 2018. \url{https://datahelpdesk.worldbank.org/knowledgebase/articles/114942-what-is-the-difference-between-current-and-constan}. Accessed 6 Apr 2018.

\leavevmode\hypertarget{ref-R353}{}%
9. R Core Team. R: A language and environment for statistical computing. Vienna, Austria: R Foundation for Statistical Computing; 2019. \url{https://www.R-project.org/}.

\end{document}


